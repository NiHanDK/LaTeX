%Dokumentklassen s�rger for at s�tte et standard layout op, med hensyn til marginer, fontst�rrelser, ekstra makroer og lignende. Det er det man m�ske ville kalde en basisskabelon
\documentclass[a4paper]{article}

%Aktivere danske og enkelsek orddelingsm�nstre
\usepackage[english, danish]{babel}

%Den f�rste pakke i oversigten s�rger for, at vi m� skrive �, � og � (og andre specialtegn) direkte i kildekoden uden at skulle anvende specialkonstruktioner
%latin1 = ISO-8859-1
\usepackage[latin1]{inputenc}

%Hj�lper med orddeling ved �, � og �.
\usepackage[T1]{fontenc}

\usepackage[top=3cm,bottom=3cm,left=3cm,right=3cm]{geometry}

%Lav indholdsfortegnelse clickable og fjern border
\usepackage{hyperref}
\hypersetup{
    colorlinks,
    citecolor=black,
    filecolor=black,
    linkcolor=black,
    urlcolor=black
}

%Til listing af programmerings kode
\usepackage{listings}
\usepackage{color}
\usepackage{xcolor}

\lstdefinestyle{defaultLayout} {
	basicstyle=\footnotesize,  				% The text size of the fonts that are used for the code
	numbers=left,                   		% Location of the line-numbers
	numberstyle=\footnotesize,  			% Style used for the line-numbers
	numberstyle=\color{gray},
	stepnumber=1,                   		% Step betweek line-numbers
	numbersep=6pt,                  		% Distance between code and linenumbers
	backgroundcolor=\color{white},      	% Background color. this will need the package usepackage{color}
	showspaces=false,               		% Real spaces instead of an _ looking thing
	showstringspaces=false,         		% Real spaces instead of an _ looking thing
	showtabs=false,                 		% Shows real tabs
	frame=single,                   		% Frame around the code
	rulecolor=\color{black},        		% Sets the color of the frame, instedad of changing when linebreak
	tabsize=2,                      		% Tabsize
	captionpos=b,                   		% Caption position to bottom
	breaklines=true,                		% Automatic line breaking
	breakatwhitespace=false,        		% Automatic breaks should only happen at whitespace
	title=\lstname,                  		% show the filename of files included with lstinputlisting;
                                  			% also try caption instead of title
}